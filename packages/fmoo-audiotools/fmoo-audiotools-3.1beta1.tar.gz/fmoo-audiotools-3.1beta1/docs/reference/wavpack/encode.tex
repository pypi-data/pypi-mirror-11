%Copyright (C) 2007-2015  Brian Langenberger
%This work is licensed under the
%Creative Commons Attribution-Share Alike 3.0 United States License.
%To view a copy of this license, visit
%http://creativecommons.org/licenses/by-sa/3.0/us/ or send a letter to
%Creative Commons,
%171 Second Street, Suite 300,
%San Francisco, California, 94105, USA.

\section{WavPack Encoding}

{\relsize{-1}
  \input{wavpack/algorithms/encode_wavpack}
}

\clearpage

\subsection{Determine Block Split}
\label{wavpack:block_split}
\ALGORITHM{input stream's channel assignment}{number of blocks per set, list of channel counts per block}
\SetKwData{BLOCKCOUNT}{block count}
\SetKwData{BLOCKCHANNELS}{block channels}
\Switch(\tcc*[f]{split channels by left/right pairs}){channel assignment}{
  \uCase{mono}{
    $\text{\BLOCKCOUNT} \leftarrow 1$\;
    $\text{\BLOCKCHANNELS} \leftarrow \texttt{[1]}$\;
  }
  \uCase{front left, front right}{
    $\text{\BLOCKCOUNT} \leftarrow 1$\;
    $\text{\BLOCKCHANNELS} \leftarrow \texttt{[2]}$\;
  }
  \uCase{front left, front right, front center}{
    $\text{\BLOCKCOUNT} \leftarrow 2$\;
    $\text{\BLOCKCHANNELS} \leftarrow \texttt{[2, 1]}$\;
  }
  \uCase{front left, front right, back left, back right}{
    $\text{\BLOCKCOUNT} \leftarrow 2$\;
    $\text{\BLOCKCHANNELS} \leftarrow \texttt{[2, 2]}$\;
  }
  \uCase{front left, front right, front center, back center}{
    $\text{\BLOCKCOUNT} \leftarrow 3$\;
    $\text{\BLOCKCHANNELS} \leftarrow \texttt{[2, 1, 1]}$\;
  }
  \uCase{front left, front right, front center, back left, back right}{
    $\text{\BLOCKCOUNT} \leftarrow 3$\;
    $\text{\BLOCKCHANNELS} \leftarrow \texttt{[2, 1, 2]}$\;
  }
  \uCase{front left, front right, front center, LFE, back left, back right}{
    $\text{\BLOCKCOUNT} \leftarrow 4$\;
    $\text{\BLOCKCHANNELS} \leftarrow \texttt{[2, 1, 1, 2]}$\;
  }
  \Other(\tcc*[f]{save them independently}){
    $\text{\BLOCKCOUNT} \leftarrow$ channel count\;
    $\text{\BLOCKCHANNELS} \leftarrow$ 1 per channel\;
  }
}
\Return \BLOCKCOUNT and \BLOCKCHANNELS
\EALGORITHM
\vskip 1ex
\par
\noindent
One could invent alternate channel splits for other obscure assignments.
WavPack's only requirement is that all channels must be in
Wave order\footnote{see page \pageref{wave_channel_assignment}}
and each block must contain 1 or 2 channels.

\begin{figure}[h]
\includegraphics{wavpack/figures/block_channels.pdf}
\end{figure}

\clearpage

\subsection{Encoding Parameters}
\label{wavpack:encoding_parameters}
{\relsize{-1}
  \input{wavpack/algorithms/encoding_parameters}
}

\clearpage

\subsection{Encoding Block}
\label{wavpack:encode_block}
{\relsize{-2}
  \input{wavpack/algorithms/encode_block}
}

\clearpage

\subsection{Calculating Maximum Magnitude}
\label{wavpack:calc_maximum_magnitude}
{\relsize{-1}
  \input{wavpack/algorithms/calculate_max_magnitude}
}

\subsection{Calculating Wasted Bits Per Sample}
\label{wavpack:calc_wasted_bps}
{\relsize{-1}
  \input{wavpack/algorithms/calculate_wasted_bps}
where the \texttt{wasted} function is defined as:
\begin{equation*}
\texttt{wasted}(x) =
\begin{cases}
\infty & \text{if } x = 0 \\
0 & \text{if } x \bmod 2 = 1 \\
1 + \texttt{wasted}(x \div 2) & \text{if } x \bmod 2 = 0 \\
\end{cases}
\end{equation*}
}

\clearpage

\subsection{Joint Stereo Conversion}
\label{wavpack:calc_joint_stereo}
\input{wavpack/algorithms/apply_joint_stereo}

\subsubsection{Joint Stereo Example}
\begin{table}[h]
{\relsize{-1}
\begin{tabular}{|r|r|r||>{$}r<{$}|>{$}r<{$}|}
$i$ & $\textsf{left}_i$ & $\textsf{right}_i$ & \textsf{mid}_i & \textsf{side}_i \\
\hline
0 & 0 & 64 & 0 - 64 = -64 & \lfloor(0 + 64) \div 2\rfloor = 32 \\
1 & 16 & 62 & 16 - 62 = -46 & \lfloor(16 + 62) \div 2\rfloor = 39 \\
2 & 31 & 56 & 31 - 56 = -25 & \lfloor(31 + 56) \div 2\rfloor = 43 \\
3 & 44 & 47 & 44 - 47 = -3 & \lfloor(44 + 47) \div 2\rfloor = 45 \\
4 & 54 & 34 & 54 - 34 = 20 & \lfloor(54 + 34) \div 2\rfloor = 44 \\
5 & 61 & 20 & 61 - 20 = 41 & \lfloor(61 + 20) \div 2\rfloor = 40 \\
6 & 64 & 4 & 64 - 4 = 60 & \lfloor(64 + 4) \div 2\rfloor = 34 \\
7 & 63 & -12 & 63 - -12 = 75 & \lfloor(63 + -12) \div 2\rfloor = 25 \\
8 & 58 & -27 & 58 - -27 = 85 & \lfloor(58 + -27) \div 2\rfloor = 15 \\
9 & 49 & -41 & 49 - -41 = 90 & \lfloor(49 + -41) \div 2\rfloor = 4 \\
\end{tabular}
}
\end{table}

\clearpage

\subsection{Writing Block Parameters}
\label{wavpack:write_block_parameters}
{\relsize{-1}
  \input{wavpack/algorithms/write_block_parameters}
}

\clearpage

\subsection{Writing Sub Block Header}
\label{wavpack:write_sub_block_header}
\input{wavpack/algorithms/write_sub_block_header}

\clearpage

%Copyright (C) 2007-2015  Brian Langenberger
%This work is licensed under the
%Creative Commons Attribution-Share Alike 3.0 United States License.
%To view a copy of this license, visit
%http://creativecommons.org/licenses/by-sa/3.0/us/ or send a letter to
%Creative Commons,
%171 Second Street, Suite 300,
%San Francisco, California, 94105, USA.

\subsection{Reading Decorrelation Terms}
\label{wavpack:read_decorrelation_terms}
\input{wavpack/algorithms/read_decorrelation_terms}

\begin{figure}[h]
  \includegraphics{wavpack/figures/decorr_terms.pdf}
\end{figure}

\clearpage

\subsubsection{Reading Decorrelation Terms Example}

\begin{figure}[h]
\includegraphics{wavpack/figures/terms_parse.pdf}
\end{figure}
\begin{center}
{\renewcommand{\arraystretch}{1.25}
\begin{tabular}{>{$}r<{$}>{$}c<{$}>{$}r<{$}|>{$}r<{$}>{$}r<{$}>{$}r<{$}}
\text{decorrelation term}_4 & \leftarrow & 23 - 5 = 18 &
\text{decorrelation delta}_4 & \leftarrow & 2 \\
\text{decorrelation term}_3 & \leftarrow & 23 - 5 = 18 &
\text{decorrelation delta}_3 & \leftarrow & 2 \\
\text{decorrelation term}_2 & \leftarrow & 7 - 5 = 2 &
\text{decorrelation delta}_2 & \leftarrow & 2 \\
\text{decorrelation term}_1 & \leftarrow & 22 - 5 = 17 &
\text{decorrelation delta}_1 & \leftarrow & 2 \\
\text{decorrelation term}_0 & \leftarrow & 8 - 5 = 3 &
\text{decorrelation delta}_0 & \leftarrow & 2 \\
\end{tabular}
\renewcommand{\arraystretch}{1.0}
}
\end{center}


\clearpage

%Copyright (C) 2007-2015  Brian Langenberger
%This work is licensed under the
%Creative Commons Attribution-Share Alike 3.0 United States License.
%To view a copy of this license, visit
%http://creativecommons.org/licenses/by-sa/3.0/us/ or send a letter to
%Creative Commons,
%171 Second Street, Suite 300,
%San Francisco, California, 94105, USA.

\subsection{Writing Decorrelation Weights}
\label{wavpack:write_decorr_weights}
\input{wavpack/algorithms/write_decorrelation_weights}
\begin{figure}[h]
  \includegraphics{wavpack/figures/decorr_weights.pdf}
\end{figure}
\clearpage
For example, given the decorrelation weight values:
\begin{table}[h]
\begin{tabular}{rrrrr}
$p$ & $\textsf{weight}_{p~0}$ & $\textsf{weight}_{p~1}$ &
$\texttt{store\_weight}(\textsf{weight}_{p~0})$ &
$\texttt{store\_weight}(\textsf{weight}_{p~1})$ \\
\hline
0 & 16 & 24 & 2 & 3 \\
1 & 48 & 48 & 6 & 6 \\
2 & 32 & 32 & 4 & 4 \\
3 & 48 & 48 & 6 & 6 \\
4 & 48 & 48 & 6 & 6 \\
\end{tabular}
\end{table}
\par
\noindent
the decorrelation weights subframe is written as:
\begin{figure}[h]
\includegraphics{wavpack/figures/decorr_weights_parse.pdf}
\end{figure}


\clearpage

%Copyright (C) 2007-2015  Brian Langenberger
%This work is licensed under the
%Creative Commons Attribution-Share Alike 3.0 United States License.
%To view a copy of this license, visit
%http://creativecommons.org/licenses/by-sa/3.0/us/ or send a letter to
%Creative Commons,
%171 Second Street, Suite 300,
%San Francisco, California, 94105, USA.

\subsection{Writing Decorrelation Samples}
\label{wavpack:write_decorr_samples}
{\relsize{-1}
  \input{wavpack/algorithms/write_decorrelation_samples}
}

\begin{figure}[h]
  \includegraphics{wavpack/figures/decorr_samples.pdf}
\end{figure}

\clearpage

\subsubsection{The wv\_log2 Function}
\label{wavpack:encode_wv_log2}
\SetKwFunction{LOG}{wlog}
{\relsize{-1}
  \input{wavpack/algorithms/encode_wv_log2}
}

\clearpage

where \texttt{wlog} is defined from the following table:
\par
\noindent
{\relsize{-3}\ttfamily
\begin{tabular}{|r|r|r|r|r|r|r|r|r|r|r|r|r|r|r|r|r|}
\hline
& \texttt{0x?0} & \texttt{0x?1} & \texttt{0x?2} & \texttt{0x?3} & \texttt{0x?4} & \texttt{0x?5} & \texttt{0x?6} & \texttt{0x?7} & \texttt{0x?8} & \texttt{0x?9} & \texttt{0x?A} & \texttt{0x?B} & \texttt{0x?C} & \texttt{0x?D} & \texttt{0x?E} & \texttt{0x?F} \\
\hline
\texttt{0x0?} & 0 & 1 & 3 & 4 & 6 & 7 & 9 & 10 & 11 & 13 & 14 & 16 & 17 & 18 & 20 & 21 \\
\texttt{0x1?} & 22 & 24 & 25 & 26 & 28 & 29 & 30 & 32 & 33 & 34 & 36 & 37 & 38 & 40 & 41 & 42 \\
\texttt{0x2?} & 44 & 45 & 46 & 47 & 49 & 50 & 51 & 52 & 54 & 55 & 56 & 57 & 59 & 60 & 61 & 62 \\
\texttt{0x3?} & 63 & 65 & 66 & 67 & 68 & 69 & 71 & 72 & 73 & 74 & 75 & 77 & 78 & 79 & 80 & 81 \\
\texttt{0x4?} & 82 & 84 & 85 & 86 & 87 & 88 & 89 & 90 & 92 & 93 & 94 & 95 & 96 & 97 & 98 & 99 \\
\texttt{0x5?} & 100 & 102 & 103 & 104 & 105 & 106 & 107 & 108 & 109 & 110 & 111 & 112 & 113 & 114 & 116 & 117 \\
\texttt{0x6?} & 118 & 119 & 120 & 121 & 122 & 123 & 124 & 125 & 126 & 127 & 128 & 129 & 130 & 131 & 132 & 133 \\
\texttt{0x7?} & 134 & 135 & 136 & 137 & 138 & 139 & 140 & 141 & 142 & 143 & 144 & 145 & 146 & 147 & 148 & 149 \\
\texttt{0x8?} & 150 & 151 & 152 & 153 & 154 & 155 & 155 & 156 & 157 & 158 & 159 & 160 & 161 & 162 & 163 & 164 \\
\texttt{0x9?} & 165 & 166 & 167 & 168 & 169 & 169 & 170 & 171 & 172 & 173 & 174 & 175 & 176 & 177 & 178 & 178 \\
\texttt{0xA?} & 179 & 180 & 181 & 182 & 183 & 184 & 185 & 185 & 186 & 187 & 188 & 189 & 190 & 191 & 192 & 192 \\
\texttt{0xB?} & 193 & 194 & 195 & 196 & 197 & 198 & 198 & 199 & 200 & 201 & 202 & 203 & 203 & 204 & 205 & 206 \\
\texttt{0xC?} & 207 & 208 & 208 & 209 & 210 & 211 & 212 & 212 & 213 & 214 & 215 & 216 & 216 & 217 & 218 & 219 \\
\texttt{0xD?} & 220 & 220 & 221 & 222 & 223 & 224 & 224 & 225 & 226 & 227 & 228 & 228 & 229 & 230 & 231 & 231 \\
\texttt{0xE?} & 232 & 233 & 234 & 234 & 235 & 236 & 237 & 238 & 238 & 239 & 240 & 241 & 241 & 242 & 243 & 244 \\
\texttt{0xF?} & 244 & 245 & 246 & 247 & 247 & 248 & 249 & 249 & 250 & 251 & 252 & 252 & 253 & 254 & 255 & 255 \\
\hline
\end{tabular}
}

\subsubsection{Writing Decorrelation Samples Example}
Given a 2 channel subframe with 5 correlation passes containing
the following correlation samples:
\begin{table}[h]
\begin{tabular}{rrrr}
pass $p$ & $\text{term}_p$ & $\text{sample}_{p~0~s}$ & $\text{sample}_{p~1~s}$ \\
\hline
0 & 3 & \texttt{[62, 68, 71]} & \texttt{[0, 10, 18]} \\
1 & 17 & \texttt{[72, 68]} & \texttt{[11, 1]} \\
2 & 2 & \texttt{[73, 77]} & \texttt{[1, 12]} \\
3 & 18 & \texttt{[79, 75]} & \texttt{[13, 2]} \\
4 & 18 & \texttt{[84, 80]} & \texttt{[14, 3]} \\
\end{tabular}
\end{table}
\par
\noindent
$\text{sample}_{4~0~0}$ (pass 4, channel 0, sample 0) is encoded as:
\begin{align*}
a &\leftarrow |84| + \lfloor|84| \div 2 ^ 9\rfloor = 84 \\
c &\leftarrow \lfloor\log_2(84)\rfloor + 1 = 7 \\
value &\leftarrow 7 \times 2 ^ 8 + \texttt{wlog}((84 \times 2 ^ 2) \bmod 256) \\
&\leftarrow 1792 + \texttt{wlog}(\texttt{0x50}) = 1892 = \texttt{0x764}
\end{align*}
\par
\noindent
and the entire sub block is written as:
\begin{figure}[h]
\includegraphics{wavpack/figures/decorr_samples_encode.pdf}
\end{figure}


\clearpage

%Copyright (C) 2007-2015  Brian Langenberger
%This work is licensed under the
%Creative Commons Attribution-Share Alike 3.0 United States License.
%To view a copy of this license, visit
%http://creativecommons.org/licenses/by-sa/3.0/us/ or send a letter to
%Creative Commons,
%171 Second Street, Suite 300,
%San Francisco, California, 94105, USA.

\subsection{Writing Entropy Variables}
\label{wavpack:write_entropy}
\input{wavpack/algorithms/write_entropy_variables}

\begin{figure}[h]
  \includegraphics{wavpack/figures/entropy_vars.pdf}
\end{figure}

\clearpage

\subsubsection{Writing Entropy Variables Example}

\begin{table}[h]
{\relsize{-2}
  \renewcommand{\arraystretch}{1.5}
\begin{tabular}{r|>{$}r<{$}|>{$}r<{$}|>{$}r<{$}}
  $\text{entropy}_{c~i}$ & $a$ & $c$ & \text{value}_{c~i} \\
  \hline
  118 &
  |118| + \lfloor |118| \div 2 ^ 9\rfloor = 118 &
  \lfloor\log_2(118)\rfloor + 1 = 7 &
  7 \times 2 ^ 8 + \texttt{wlog}((118 \times 2 ^ {9 - 7}) \bmod 256) = 2018 \\
  194 &
  |194| + \lfloor |194| \div 2 ^ 9\rfloor = 194 &
  \lfloor\log_2(194)\rfloor + 1 = 8 &
  8 \times 2 ^ 8 + \texttt{wlog}((194 \times 2 ^ {9 - 8}) \bmod 256) = 2202 \\
  322 &
  |322| + \lfloor |322| \div 2 ^ 9\rfloor = 322 &
  \lfloor\log_2(322)\rfloor + 1 = 9 &
  9 \times 2 ^ 8 + \LOG(\lfloor 322 \div 2 ^ {9 - 9}\rfloor \bmod 256) = 2389 \\
  \hline
  118 &
  |118| + \lfloor |118| \div 2 ^ 9\rfloor = 118 &
  \lfloor\log_2(118)\rfloor + 1 = 7 &
  7 \times 2 ^ 8 + \texttt{wlog}((118 \times 2 ^ {9 - 7}) \bmod 256) = 2018 \\
  176 &
  |176| + \lfloor |176| \div 2 ^ 9\rfloor = 176 &
  \lfloor\log_2(176)\rfloor + 1 = 8 &
  8 \times 2 ^ 8 + \texttt{wlog}((176 \times 2 ^ {9 - 8}) \bmod 256) = 2166 \\
  212 &
  |212| + \lfloor |212| \div 2 ^ 9\rfloor = 212 &
  \lfloor\log_2(212)\rfloor + 1 = 8 &
  8 \times 2 ^ 8 + \texttt{wlog}((212 \times 2 ^ {9 - 8}) \bmod 256) = 2234 \\
\end{tabular}
}
\end{table}
\begin{figure}[h]
  \includegraphics{wavpack/figures/entropy_vars_parse.pdf}
\end{figure}


\clearpage

%Copyright (C) 2007-2015  Brian Langenberger
%This work is licensed under the
%Creative Commons Attribution-Share Alike 3.0 United States License.
%To view a copy of this license, visit
%http://creativecommons.org/licenses/by-sa/3.0/us/ or send a letter to
%Creative Commons,
%171 Second Street, Suite 300,
%San Francisco, California, 94105, USA.

\subsection{Channel Correlation}
\label{wavpack:correlate_channels}
{\relsize{-1}
  \input{wavpack/algorithms/correlate_channels}
}

\clearpage

\subsection{1 Channel Correlation Pass}
\label{wavpack:correlate_1ch}
{\relsize{-1}
  \input{wavpack/algorithms/correlate_1ch}
}

\clearpage

\subsection{2 Channel Correlation Passes}
\label{wavpack:correlate_2ch}
{\relsize{-1}
  \input{wavpack/algorithms/correlate_2ch}
}

\clearpage

\subsubsection{2 Channel Correlation Pass, Term -1}
\label{wavpack:correlate_2ch_1}
{\relsize{-1}
  \input{wavpack/algorithms/correlate_2ch_1}
}

\subsubsection{2 Channel Correlation Pass, Term -2}
\label{wavpack:correlate_2ch_2}
{\relsize{-1}
  \input{wavpack/algorithms/correlate_2ch_2}
}

\clearpage

\subsubsection{2 Channel Correlation Pass, Term -3}
\label{wavpack:correlate_2ch_3}
{\relsize{-1}
  \input{wavpack/algorithms/correlate_2ch_3}
}

\clearpage

\subsection{Channel Correlation Example}
\begin{figure}[h]
{\relsize{-1}
  \subfloat{
    \begin{tabular}{|r|r|r|}
      \multicolumn{3}{c}{Correlation Terms} \\
      \hline
      $p$ & $\textsf{term}_p$ & $\textsf{delta}_p$ \\
      \hline
      0 & 3 & 2 \\
      1 & 17 & 2 \\
      2 & 2 & 2 \\
      3 & 18 & 2 \\
      4 & 18 & 2 \\
      \hline
    \end{tabular}
  }
  \subfloat{
    \begin{tabular}{|r|r|r|}
      \multicolumn{3}{c}{Correlation Weights} \\
      \hline
      $p$ & $\textsf{weight}_{p~0}$ & $\textsf{weight}_{p~1}$ \\
      \hline
      0 & 16 & 24 \\
      1 & 48 & 48 \\
      2 & 32 & 32 \\
      3 & 48 & 48 \\
      4 & 48 & 48 \\
      \hline
    \end{tabular}
  }
  \subfloat{
    \begin{tabular}{|r|r|r|}
      \multicolumn{3}{c}{Correlation Samples} \\
      \hline
      $p$ & $\textsf{sample}_{p~0~s}$ & $\textsf{sample}_{p~1~s}$ \\
      \hline
      0 & \texttt{[0, 0, 0]} & \texttt{[0, 0, 0]} \\
      1 & \texttt{[0, 0]} & \texttt{[0, 0]} \\
      2 & \texttt{[0, 0]} & \texttt{[0, 0]} \\
      3 & \texttt{[0, 0]} & \texttt{[0, 0]} \\
      4 & \texttt{[-73, -78]} & \texttt{[28, 26]} \\
      \hline
    \end{tabular}
  }
}
\end{figure}
\par
\noindent
we combine them into a single set of arguments for each correlation pass:
\begin{table}[h]
{\relsize{-1}
  \begin{tabular}{|r|r|r|r|r|r|}
    \hline
    & $\textbf{pass}_0$ & $\textbf{pass}_1$ & $\textbf{pass}_2$ &
    $\textbf{pass}_3$ & $\textbf{pass}_3$ \\
    \hline
    $\textsf{term}_p$ & 18 & 18 & 2 & 17 & 3 \\
    $\textsf{delta}_p$ & 2 & 2 & 2 & 2 & 2 \\
    $\textsf{weight}_{p~0}$ & 48 & 48 & 32 & 48 & 16 \\
    $\textsf{sample}_{p~0~s}$ & \texttt{[-73, -78]} & \texttt{[0, 0]} &
    \texttt{[0, 0]} & \texttt{[0, 0]} & \texttt{[0, 0, 0]} \\
    $\textsf{weight}_{p~1}$ & 48 & 48 & 32 & 48 & 24 \\
    $\textsf{sample}_{p~1~s}$ & \texttt{[28, 26]} & \texttt{[0, 0]} &
    \texttt{[0, 0]} & \texttt{[0, 0]} & \texttt{[0, 0, 0]} \\
    \hline
  \end{tabular}
}
\end{table}
\par
\noindent
which we apply to the residuals from the bitstream sub-block:
\par
\noindent
{\relsize{-1}
  \begin{tabular}{|r|r|r|r|r|r|}
    \hline
    $\textsf{channel}_{0~i}$ &
    after $\textbf{pass}_0$ &
    after $\textbf{pass}_1$ &
    after $\textbf{pass}_2$ &
    after $\textbf{pass}_3$ &
    after $\textbf{pass}_4$ \\
    \hline
    -64 & -61 & -61 & -61 & -61 & -61 \\
    -46 & -43 & -39 & -39 & -33 & -33 \\
    -25 & -23 & -21 & -19 & -18 & -18 \\
    -3 & -2 & -1 & 0 & 0 & 1 \\
    20 & 20 & 20 & 21 & 20 & 20 \\
    41 & 39 & 37 & 37 & 35 & 35 \\
    60 & 57 & 54 & 53 & 50 & 50 \\
    75 & 71 & 67 & 66 & 62 & 62 \\
    85 & 80 & 75 & 73 & 68 & 68 \\
    90 & 84 & 79 & 77 & 72 & 71 \\
    \hline
    \hline
    $\textsf{channel}_{1~i}$ &
    after $\textbf{pass}_0$ &
    after $\textbf{pass}_1$ &
    after $\textbf{pass}_2$ &
    after $\textbf{pass}_3$ &
    after $\textbf{pass}_4$ \\
    \hline
    32 & 31 & 31 & 31 & 31 & 31 \\
    39 & 37 & 35 & 35 & 32 & 32 \\
    43 & 41 & 39 & 38 & 36 & 36 \\
    45 & 43 & 41 & 40 & 38 & 37 \\
    44 & 41 & 39 & 38 & 36 & 35 \\
    40 & 38 & 36 & 34 & 32 & 31 \\
    34 & 32 & 30 & 28 & 26 & 25 \\
    25 & 23 & 21 & 20 & 19 & 18 \\
    15 & 14 & 13 & 12 & 11 & 10 \\
    4 & 3 & 2 & 1 & 1 & 0 \\
    \hline
  \end{tabular}
}
\par
\noindent
Resulting in final correlated samples:
\newline
\begin{tabular}{rr}
$\textsf{residual}_0$ : & \texttt{[-61,~-33,~-18,~~1,~20,~35,~50,~62,~68,~71]} \\
$\textsf{residual}_1$ : & \texttt{[~31,~~32,~~36,~37,~35,~31,~25,~18,~10,~~0]} \\
\end{tabular}

\clearpage

{\relsize{-2}
\begin{tabular}{r||r|>{$}r<{$}|>{$}r<{$}|>{$}r<{$}|>{$}r<{$}}
& $i$ & \textsf{uncorrelated}_i & \textsf{temp}_i & \textsf{correlated}_i & \textsf{weight}_{i + 1} \\
\hline
%%START
\multirow{10}{1em}{\begin{sideways}$\textbf{pass}_0$ - term 18\end{sideways}}
& 0 & -64 &
\lfloor(3 \times -73 + 78) \div 2\rfloor = -71 &
-64 - \lfloor(48 \times -71 + 2 ^ 9) \div 2 ^ {10}\rfloor = -61 &
48 + 2 = 50
\\
& 1 & -46 &
\lfloor(3 \times -64 + 73) \div 2\rfloor = -60 &
-46 - \lfloor(50 \times -60 + 2 ^ 9) \div 2 ^ {10}\rfloor = -43 &
50 + 2 = 52
\\
& 2 & -25 &
\lfloor(3 \times -46 + 64) \div 2\rfloor = -37 &
-25 - \lfloor(52 \times -37 + 2 ^ 9) \div 2 ^ {10}\rfloor = -23 &
52 + 2 = 54
\\
& 3 & -3 &
\lfloor(3 \times -25 + 46) \div 2\rfloor = -15 &
-3 - \lfloor(54 \times -15 + 2 ^ 9) \div 2 ^ {10}\rfloor = -2 &
54 + 2 = 56
\\
& 4 & 20 &
\lfloor(3 \times -3 + 25) \div 2\rfloor = 8 &
20 - \lfloor(56 \times 8 + 2 ^ 9) \div 2 ^ {10}\rfloor = 20 &
56 + 2 = 58
\\
& 5 & 41 &
\lfloor(3 \times 20 + 3) \div 2\rfloor = 31 &
41 - \lfloor(58 \times 31 + 2 ^ 9) \div 2 ^ {10}\rfloor = 39 &
58 + 2 = 60
\\
& 6 & 60 &
\lfloor(3 \times 41 - 20) \div 2\rfloor = 51 &
60 - \lfloor(60 \times 51 + 2 ^ 9) \div 2 ^ {10}\rfloor = 57 &
60 + 2 = 62
\\
& 7 & 75 &
\lfloor(3 \times 60 - 41) \div 2\rfloor = 69 &
75 - \lfloor(62 \times 69 + 2 ^ 9) \div 2 ^ {10}\rfloor = 71 &
62 + 2 = 64
\\
& 8 & 85 &
\lfloor(3 \times 75 - 60) \div 2\rfloor = 82 &
85 - \lfloor(64 \times 82 + 2 ^ 9) \div 2 ^ {10}\rfloor = 80 &
64 + 2 = 66
\\
& 9 & 90 &
\lfloor(3 \times 85 - 75) \div 2\rfloor = 90 &
90 - \lfloor(66 \times 90 + 2 ^ 9) \div 2 ^ {10}\rfloor = 84 &
66 + 2 = 68
\\
\hline
\hline
\multirow{10}{1em}{\begin{sideways}$\textbf{pass}_1$ - term 18\end{sideways}}
& 0 & -61 &
\lfloor(3 \times 0 - 0) \div 2\rfloor = 0 &
-61 - \lfloor(48 \times 0 + 2 ^ 9) \div 2 ^ {10}\rfloor = -61 &
48 + 0 = 48
\\
& 1 & -43 &
\lfloor(3 \times -61 - 0) \div 2\rfloor = -92 &
-43 - \lfloor(48 \times -92 + 2 ^ 9) \div 2 ^ {10}\rfloor = -39 &
48 + 2 = 50
\\
& 2 & -23 &
\lfloor(3 \times -43 + 61) \div 2\rfloor = -34 &
-23 - \lfloor(50 \times -34 + 2 ^ 9) \div 2 ^ {10}\rfloor = -21 &
50 + 2 = 52
\\
& 3 & -2 &
\lfloor(3 \times -23 + 43) \div 2\rfloor = -13 &
-2 - \lfloor(52 \times -13 + 2 ^ 9) \div 2 ^ {10}\rfloor = -1 &
52 + 2 = 54
\\
& 4 & 20 &
\lfloor(3 \times -2 + 23) \div 2\rfloor = 8 &
20 - \lfloor(54 \times 8 + 2 ^ 9) \div 2 ^ {10}\rfloor = 20 &
54 + 2 = 56
\\
& 5 & 39 &
\lfloor(3 \times 20 + 2) \div 2\rfloor = 31 &
39 - \lfloor(56 \times 31 + 2 ^ 9) \div 2 ^ {10}\rfloor = 37 &
56 + 2 = 58
\\
& 6 & 57 &
\lfloor(3 \times 39 - 20) \div 2\rfloor = 48 &
57 - \lfloor(58 \times 48 + 2 ^ 9) \div 2 ^ {10}\rfloor = 54 &
58 + 2 = 60
\\
& 7 & 71 &
\lfloor(3 \times 57 - 39) \div 2\rfloor = 66 &
71 - \lfloor(60 \times 66 + 2 ^ 9) \div 2 ^ {10}\rfloor = 67 &
60 + 2 = 62
\\
& 8 & 80 &
\lfloor(3 \times 71 - 57) \div 2\rfloor = 78 &
80 - \lfloor(62 \times 78 + 2 ^ 9) \div 2 ^ {10}\rfloor = 75 &
62 + 2 = 64
\\
& 9 & 84 &
\lfloor(3 \times 80 - 71) \div 2\rfloor = 84 &
84 - \lfloor(64 \times 84 + 2 ^ 9) \div 2 ^ {10}\rfloor = 79 &
64 + 2 = 66
\\
\hline
\hline
\multirow{10}{1em}{\begin{sideways}$\textbf{pass}_2$ - term 2\end{sideways}}
& 0 & -61 & &
-61 - \lfloor(32 \times 0 + 2 ^ 9) \div 2 ^ {10}\rfloor = -61 &
32 + 0 = 32
\\
& 1 & -39 & &
-39 - \lfloor(32 \times 0 + 2 ^ 9) \div 2 ^ {10}\rfloor = -39 &
32 + 0 = 32
\\
& 2 & -21 & &
-21 - \lfloor(32 \times -61 + 2 ^ 9) \div 2 ^ {10}\rfloor = -19 &
32 + 2 = 34
\\
& 3 & -1 & &
-1 - \lfloor(34 \times -39 + 2 ^ 9) \div 2 ^ {10}\rfloor = 0 &
34 + 0 = 34
\\
& 4 & 20 & &
20 - \lfloor(34 \times -21 + 2 ^ 9) \div 2 ^ {10}\rfloor = 21 &
34 - 2 = 32
\\
& 5 & 37 & &
37 - \lfloor(32 \times -1 + 2 ^ 9) \div 2 ^ {10}\rfloor = 37 &
32 - 2 = 30
\\
& 6 & 54 & &
54 - \lfloor(30 \times 20 + 2 ^ 9) \div 2 ^ {10}\rfloor = 53 &
30 + 2 = 32
\\
& 7 & 67 & &
67 - \lfloor(32 \times 37 + 2 ^ 9) \div 2 ^ {10}\rfloor = 66 &
32 + 2 = 34
\\
& 8 & 75 & &
75 - \lfloor(34 \times 54 + 2 ^ 9) \div 2 ^ {10}\rfloor = 73 &
34 + 2 = 36
\\
& 9 & 79 & &
79 - \lfloor(36 \times 67 + 2 ^ 9) \div 2 ^ {10}\rfloor = 77 &
36 + 2 = 38
\\
\hline
\hline
\multirow{10}{1em}{\begin{sideways}$\textbf{pass}_3$ - term 17\end{sideways}}
& 0 & -61 &
2 \times 0 - 0 = 0 &
-61 - \lfloor(48 \times 0 + 2 ^ 9) \div 2 ^ {10}\rfloor = -61 &
48 + 0 = 48
\\
& 1 & -39 &
2 \times -61 - 0 = -122 &
-39 - \lfloor(48 \times -122 + 2 ^ 9) \div 2 ^ {10}\rfloor = -33 &
48 + 2 = 50
\\
& 2 & -19 &
2 \times -39 + 61 = -17 &
-19 - \lfloor(50 \times -17 + 2 ^ 9) \div 2 ^ {10}\rfloor = -18 &
50 + 2 = 52
\\
& 3 & 0 &
2 \times -19 + 39 = 1 &
0 - \lfloor(52 \times 1 + 2 ^ 9) \div 2 ^ {10}\rfloor = 0 &
52 + 0 = 52
\\
& 4 & 21 &
2 \times 0 + 19 = 19 &
21 - \lfloor(52 \times 19 + 2 ^ 9) \div 2 ^ {10}\rfloor = 20 &
52 + 2 = 54
\\
& 5 & 37 &
2 \times 21 - 0 = 42 &
37 - \lfloor(54 \times 42 + 2 ^ 9) \div 2 ^ {10}\rfloor = 35 &
54 + 2 = 56
\\
& 6 & 53 &
2 \times 37 - 21 = 53 &
53 - \lfloor(56 \times 53 + 2 ^ 9) \div 2 ^ {10}\rfloor = 50 &
56 + 2 = 58
\\
& 7 & 66 &
2 \times 53 - 37 = 69 &
66 - \lfloor(58 \times 69 + 2 ^ 9) \div 2 ^ {10}\rfloor = 62 &
58 + 2 = 60
\\
& 8 & 73 &
2 \times 66 - 53 = 79 &
73 - \lfloor(60 \times 79 + 2 ^ 9) \div 2 ^ {10}\rfloor = 68 &
60 + 2 = 62
\\
& 9 & 77 &
2 \times 73 - 66 = 80 &
77 - \lfloor(62 \times 80 + 2 ^ 9) \div 2 ^ {10}\rfloor = 72 &
62 + 2 = 64
\\
\hline
\hline
\multirow{10}{1em}{\begin{sideways}$\textbf{pass}_4$ - term 3\end{sideways}}
& 0 & -61 & &
-61 - \lfloor(16 \times 0 + 2 ^ 9) \div 2 ^ {10}\rfloor = -61 &
16 + 0 = 16
\\
& 1 & -33 & &
-33 - \lfloor(16 \times 0 + 2 ^ 9) \div 2 ^ {10}\rfloor = -33 &
16 + 0 = 16
\\
& 2 & -18 & &
-18 - \lfloor(16 \times 0 + 2 ^ 9) \div 2 ^ {10}\rfloor = -18 &
16 + 0 = 16
\\
& 3 & 0 & &
0 - \lfloor(16 \times -61 + 2 ^ 9) \div 2 ^ {10}\rfloor = 1 &
16 - 2 = 14
\\
& 4 & 20 & &
20 - \lfloor(14 \times -33 + 2 ^ 9) \div 2 ^ {10}\rfloor = 20 &
14 - 2 = 12
\\
& 5 & 35 & &
35 - \lfloor(12 \times -18 + 2 ^ 9) \div 2 ^ {10}\rfloor = 35 &
12 - 2 = 10
\\
& 6 & 50 & &
50 - \lfloor(10 \times 0 + 2 ^ 9) \div 2 ^ {10}\rfloor = 50 &
10 + 0 = 10
\\
& 7 & 62 & &
62 - \lfloor(10 \times 20 + 2 ^ 9) \div 2 ^ {10}\rfloor = 62 &
10 + 2 = 12
\\
& 8 & 68 & &
68 - \lfloor(12 \times 35 + 2 ^ 9) \div 2 ^ {10}\rfloor = 68 &
12 + 2 = 14
\\
& 9 & 72 & &
72 - \lfloor(14 \times 50 + 2 ^ 9) \div 2 ^ {10}\rfloor = 71 &
14 + 2 = 16
\\
%%END
\end{tabular}
}
\begin{center}
$\text{channel}_0$ correlation passes
\end{center}


\clearpage

%Copyright (C) 2007-2015  Brian Langenberger
%This work is licensed under the
%Creative Commons Attribution-Share Alike 3.0 United States License.
%To view a copy of this license, visit
%http://creativecommons.org/licenses/by-sa/3.0/us/ or send a letter to
%Creative Commons,
%171 Second Street, Suite 300,
%San Francisco, California, 94105, USA.

\subsection{Writing Bitstream}
\label{wavpack:write_bitstream}
\input{wavpack/algorithms/write_bitstream}

\clearpage

\subsection{Encoding Bitstream}
\label{wavpack:encode_bitstream}
{\relsize{-1}
  \input{wavpack/algorithms/encode_bitstream}
}

\clearpage

\subsubsection{Writing Modified Elias Gamma Code}
\label{wavpack:write_egc}
\input{wavpack/algorithms/write_egc}

\begin{figure}[h]
  \includegraphics{wavpack/figures/residuals.pdf}
\end{figure}

\clearpage

\subsubsection{\texttt{encode\_residual} Function}
\label{wavpack:encode_residual}
{\relsize{-1}
  \input{wavpack/algorithms/encode_residual}
}

\clearpage

\subsubsection{\texttt{flush} Function}
\label{wavpack:flush_residual}
{\relsize{-1}
  \input{wavpack/algorithms/flush_residual}
}

\clearpage

\subsubsection{\texttt{encode\_zeroes} Function}
\label{wavpack:encode_zeroes}
{\relsize{-1}
  \input{wavpack/algorithms/encode_zeroes}
}

\clearpage


\subsubsection{Residual Encoding Example}
{\relsize{-1}
  Given the residuals:
  \newline
  \begin{tabular}{rr}
    channel 0 residuals : & \texttt{[-61,~-33,~-18,~~1,~20,~35,~50,~62,~68,~71]}\\
    channel 1 residuals : & \texttt{[~31,~~32,~~36,~37,~35,~31,~25,~18,~10,~~0]}\\
  \end{tabular}
  \newline
  And entropies:
  \newline
  \begin{tabular}{rr}
    channel 0 entropy : & \texttt{[118, 194, 322]} \\
    channel 1 entropy : & \texttt{[118, 176, 212]} \\
  \end{tabular}
}

\begin{table}[h]
{\relsize{-2}
\begin{tabular}{|>{$}r<{$}||>{$}r<{$}|>{$}r<{$}|>{$}r<{$}|>{$}r<{$}|>{$}r<{$}|>{$}r<{$}|}
i & r_i & \text{unsigned}_i &\text{sign}_i &
%% \text{median}_{c~0} & \text{median}_{c~1} & \text{median}_{c~2} &
m_i & \text{offset}_i & \text{add}_i \\
\hline
0 & -61 &
60 & 1 &
%% \left\lfloor\frac{118}{2 ^ 4}\right\rfloor + 1 = 8 & \left\lfloor\frac{194}{2 ^ 4}\right\rfloor + 1 = 13 & \left\lfloor\frac{322}{2 ^ 4}\right\rfloor + 1 = 21 &
3 & 60 - (8 + 13 + ((3 - 2) \times 21)) = 18 & 21 - 1 = 20
\\
1 & 31 &
31 & 0 &
%% \left\lfloor\frac{118}{2 ^ 4}\right\rfloor + 1 = 8 & \left\lfloor\frac{176}{2 ^ 4}\right\rfloor + 1 = 12 & \left\lfloor\frac{212}{2 ^ 4}\right\rfloor + 1 = 14 &
2 & 31 - (8 + 12) = 11 & 14 - 1 = 13
\\
\hline
2 & -33 &
32 & 1 &
%% \left\lfloor\frac{123}{2 ^ 4}\right\rfloor + 1 = 8 & \left\lfloor\frac{214}{2 ^ 4}\right\rfloor + 1 = 14 & \left\lfloor\frac{377}{2 ^ 4}\right\rfloor + 1 = 24 &
2 & 32 - (8 + 14) = 10 & 24 - 1 = 23
\\
3 & 32 &
32 & 0 &
%% \left\lfloor\frac{123}{2 ^ 4}\right\rfloor + 1 = 8 & \left\lfloor\frac{191}{2 ^ 4}\right\rfloor + 1 = 12 & \left\lfloor\frac{198}{2 ^ 4}\right\rfloor + 1 = 13 &
2 & 32 - (8 + 12) = 12 & 13 - 1 = 12
\\
\hline
4 & -18 &
17 & 1 &
%% \left\lfloor\frac{128}{2 ^ 4}\right\rfloor + 1 = 9 & \left\lfloor\frac{234}{2 ^ 4}\right\rfloor + 1 = 15 & \left\lfloor\frac{353}{2 ^ 4}\right\rfloor + 1 = 23 &
1 & 17 - 9 = 8 & 15 - 1 = 14
\\
5 & 36 &
36 & 0 &
%% \left\lfloor\frac{128}{2 ^ 4}\right\rfloor + 1 = 9 & \left\lfloor\frac{206}{2 ^ 4}\right\rfloor + 1 = 13 & \left\lfloor\frac{184}{2 ^ 4}\right\rfloor + 1 = 12 &
3 & 36 - (9 + 13 + ((3 - 2) \times 12)) = 2 & 12 - 1 = 11
\\
\hline
6 & 1 &
1 & 0 &
%% \left\lfloor\frac{138}{2 ^ 4}\right\rfloor + 1 = 9 & \left\lfloor\frac{226}{2 ^ 4}\right\rfloor + 1 = 15 & \left\lfloor\frac{353}{2 ^ 4}\right\rfloor + 1 = 23 &
0 & 1 & 9 - 1 = 8
\\
7 & 37 &
37 & 0 &
%% \left\lfloor\frac{138}{2 ^ 4}\right\rfloor + 1 = 9 & \left\lfloor\frac{226}{2 ^ 4}\right\rfloor + 1 = 15 & \left\lfloor\frac{214}{2 ^ 4}\right\rfloor + 1 = 14 &
2 & 37 - (9 + 15) = 13 & 14 - 1 = 13
\\
\hline
8 & 20 &
20 & 0 &
%% \left\lfloor\frac{134}{2 ^ 4}\right\rfloor + 1 = 9 & \left\lfloor\frac{226}{2 ^ 4}\right\rfloor + 1 = 15 & \left\lfloor\frac{353}{2 ^ 4}\right\rfloor + 1 = 23 &
1 & 20 - 9 = 11 & 15 - 1 = 14
\\
9 & 35 &
35 & 0 &
%% \left\lfloor\frac{148}{2 ^ 4}\right\rfloor + 1 = 10 & \left\lfloor\frac{246}{2 ^ 4}\right\rfloor + 1 = 16 & \left\lfloor\frac{200}{2 ^ 4}\right\rfloor + 1 = 13 &
2 & 35 - (10 + 16) = 9 & 13 - 1 = 12
\\
\hline
10 & 35 &
35 & 0 &
%% \left\lfloor\frac{144}{2 ^ 4}\right\rfloor + 1 = 10 & \left\lfloor\frac{218}{2 ^ 4}\right\rfloor + 1 = 14 & \left\lfloor\frac{353}{2 ^ 4}\right\rfloor + 1 = 23 &
2 & 35 - (10 + 14) = 11 & 23 - 1 = 22
\\
11 & 31 &
31 & 0 &
%% \left\lfloor\frac{158}{2 ^ 4}\right\rfloor + 1 = 10 & \left\lfloor\frac{266}{2 ^ 4}\right\rfloor + 1 = 17 & \left\lfloor\frac{186}{2 ^ 4}\right\rfloor + 1 = 12 &
2 & 31 - (10 + 17) = 4 & 12 - 1 = 11
\\
\hline
12 & 50 &
50 & 0 &
%% \left\lfloor\frac{154}{2 ^ 4}\right\rfloor + 1 = 10 & \left\lfloor\frac{238}{2 ^ 4}\right\rfloor + 1 = 15 & \left\lfloor\frac{331}{2 ^ 4}\right\rfloor + 1 = 21 &
3 & 50 - (10 + 15 + ((3 - 2) \times 21)) = 4 & 21 - 1 = 20
\\
13 & 25 &
25 & 0 &
%% \left\lfloor\frac{168}{2 ^ 4}\right\rfloor + 1 = 11 & \left\lfloor\frac{291}{2 ^ 4}\right\rfloor + 1 = 19 & \left\lfloor\frac{174}{2 ^ 4}\right\rfloor + 1 = 11 &
1 & 25 - 11 = 14 & 19 - 1 = 18
\\
\hline
14 & 62 &
62 & 0 &
%% \left\lfloor\frac{164}{2 ^ 4}\right\rfloor + 1 = 11 & \left\lfloor\frac{258}{2 ^ 4}\right\rfloor + 1 = 17 & \left\lfloor\frac{386}{2 ^ 4}\right\rfloor + 1 = 25 &
3 & 62 - (11 + 17 + ((3 - 2) \times 25)) = 9 & 25 - 1 = 24
\\
15 & 18 &
18 & 0 &
%% \left\lfloor\frac{178}{2 ^ 4}\right\rfloor + 1 = 12 & \left\lfloor\frac{281}{2 ^ 4}\right\rfloor + 1 = 18 & \left\lfloor\frac{174}{2 ^ 4}\right\rfloor + 1 = 11 &
1 & 18 - 12 = 6 & 18 - 1 = 17
\\
\hline
16 & 68 &
68 & 0 &
%% \left\lfloor\frac{174}{2 ^ 4}\right\rfloor + 1 = 11 & \left\lfloor\frac{283}{2 ^ 4}\right\rfloor + 1 = 18 & \left\lfloor\frac{451}{2 ^ 4}\right\rfloor + 1 = 29 &
3 & 68 - (11 + 18 + ((3 - 2) \times 29)) = 10 & 29 - 1 = 28
\\
17 & 10 &
10 & 0 &
%% \left\lfloor\frac{188}{2 ^ 4}\right\rfloor + 1 = 12 & \left\lfloor\frac{271}{2 ^ 4}\right\rfloor + 1 = 17 & \left\lfloor\frac{174}{2 ^ 4}\right\rfloor + 1 = 11 &
0 & 10 & 12 - 1 = 11
\\
\hline
18 & 71 &
71 & 0 &
%% \left\lfloor\frac{184}{2 ^ 4}\right\rfloor + 1 = 12 & \left\lfloor\frac{308}{2 ^ 4}\right\rfloor + 1 = 20 & \left\lfloor\frac{526}{2 ^ 4}\right\rfloor + 1 = 33 &
3 & 71 - (12 + 20 + ((3 - 2) \times 33)) = 6 & 33 - 1 = 32
\\
19 & 0 &
0 & 0 &
%% \left\lfloor\frac{184}{2 ^ 4}\right\rfloor + 1 = 12 & \left\lfloor\frac{271}{2 ^ 4}\right\rfloor + 1 = 17 & \left\lfloor\frac{174}{2 ^ 4}\right\rfloor + 1 = 11 &
0 & 0 & 12 - 1 = 11
\\
\hline
\end{tabular}
\vskip .25in
\begin{tabular}{|>{$}r<{$}||>{$}r<{$}|>{$}r<{$}|>{$}r<{$}||>{$}r<{$}|>{$}r<{$}|>{$}r<{$}|>{$}r<{$}>{$}r<{$}|}
i & m_i & \text{offset}_i & \text{add}_i & u_i & p_i & e_i & r_i & b_i \\
\hline
0 & 3 &
18 & 20 &
(3 \times 2) + 1 = 7 &
\lfloor\log_2(20)\rfloor = 4 &
2 ^ {(4 + 1)} - 20 - 1 = 11 &
(18 + 11) \div 2 = 14 &
1 \\
1 & 2 &
11 & 13 &
(2 \times 2) - 1 = 3 &
\lfloor\log_2(13)\rfloor = 3 &
2 ^ {(3 + 1)} - 13 - 1 = 2 &
(11 + 2) \div 2 = 6 &
1 \\
\hline
2 & 2 &
10 & 23 &
(2 \times 2) - 1 = 3 &
\lfloor\log_2(23)\rfloor = 4 &
2 ^ {(4 + 1)} - 23 - 1 = 8 &
(10 + 8) \div 2 = 9 &
0 \\
3 & 2 &
12 & 12 &
(2 \times 2) - 1 = 3 &
\lfloor\log_2(12)\rfloor = 3 &
2 ^ {(3 + 1)} - 12 - 1 = 3 &
(12 + 3) \div 2 = 7 &
1 \\
\hline
4 & 1 &
8 & 14 &
(1 \times 2) - 1 = 1 &
\lfloor\log_2(14)\rfloor = 3 &
2 ^ {(3 + 1)} - 14 - 1 = 1 &
(8 + 1) \div 2 = 4 &
1 \\
5 & 3 &
2 & 11 &
(3 - 1) \times 2 = 4 &
\lfloor\log_2(11)\rfloor = 3 &
2 ^ {(3 + 1)} - 11 - 1 = 4 &
2 & \\
\hline
6 & 0 &
1 & 8 &
\textit{undefined} &
\lfloor\log_2(8)\rfloor = 3 &
2 ^ {(3 + 1)} - 8 - 1 = 7 &
1 & \\
7 & 2 &
13 & 13 &
(2 \times 2) + 1 = 5 &
\lfloor\log_2(13)\rfloor = 3 &
2 ^ {(3 + 1)} - 13 - 1 = 2 &
(13 + 2) \div 2 = 7 &
1 \\
\hline
8 & 1 &
11 & 14 &
(1 \times 2) - 1 = 1 &
\lfloor\log_2(14)\rfloor = 3 &
2 ^ {(3 + 1)} - 14 - 1 = 1 &
(11 + 1) \div 2 = 6 &
0 \\
9 & 2 &
9 & 12 &
(2 \times 2) - 1 = 3 &
\lfloor\log_2(12)\rfloor = 3 &
2 ^ {(3 + 1)} - 12 - 1 = 3 &
(9 + 3) \div 2 = 6 &
0 \\
\hline
10 & 2 &
11 & 22 &
(2 \times 2) - 1 = 3 &
\lfloor\log_2(22)\rfloor = 4 &
2 ^ {(4 + 1)} - 22 - 1 = 9 &
(11 + 9) \div 2 = 10 &
0 \\
11 & 2 &
4 & 11 &
(2 \times 2) - 1 = 3 &
\lfloor\log_2(11)\rfloor = 3 &
2 ^ {(3 + 1)} - 11 - 1 = 4 &
(4 + 4) \div 2 = 4 &
0 \\
\hline
12 & 3 &
4 & 20 &
(3 \times 2) - 1 = 5 &
\lfloor\log_2(20)\rfloor = 4 &
2 ^ {(4 + 1)} - 20 - 1 = 11 &
4 & \\
13 & 1 &
14 & 18 &
(1 \times 2) - 1 = 1 &
\lfloor\log_2(18)\rfloor = 4 &
2 ^ {(4 + 1)} - 18 - 1 = 13 &
(14 + 13) \div 2 = 13 &
1 \\
\hline
14 & 3 &
9 & 24 &
(3 \times 2) - 1 = 5 &
\lfloor\log_2(24)\rfloor = 4 &
2 ^ {(4 + 1)} - 24 - 1 = 7 &
(9 + 7) \div 2 = 8 &
0 \\
15 & 1 &
6 & 17 &
(1 \times 2) - 1 = 1 &
\lfloor\log_2(17)\rfloor = 4 &
2 ^ {(4 + 1)} - 17 - 1 = 14 &
6 & \\
\hline
16 & 3 &
10 & 28 &
(3 - 1) \times 2 = 4 &
\lfloor\log_2(28)\rfloor = 4 &
2 ^ {(4 + 1)} - 28 - 1 = 3 &
(9 + 3) \div 2 = 6 &
0 \\
17 & 0 &
10 & 11 &
\textit{undefined} &
\lfloor\log_2(11)\rfloor = 3 &
2 ^ {(3 + 1)} - 11 - 1 = 4 &
(10 + 4) \div 2 = 7 &
0 \\
\hline
18 & 3 &
6 & 32 &
3 \times 2 = 6 &
\lfloor\log_2(32)\rfloor = 5 &
2 ^ {(5 + 1)} - 32 - 1 = 31 &
6 & \\
19 & 0 &
0 & 11 &
\textit{undefined} &
\lfloor\log_2(11)\rfloor = 3 &
2 ^ {(3 + 1)} - 11 - 1 = 4 &
0 & \\
\hline
\end{tabular}
}
\end{table}

\clearpage

\begin{figure}[h]
  \includegraphics[width=6in,keepaspectratio]{wavpack/figures/residuals_parse.pdf}
\end{figure}

\clearpage

\subsubsection{2nd Residual Encoding Example}
This example is more simplified to demonstrate how the \VAR{zeroes}
value propagates in two instances.
\vskip .25in
{\relsize{-2}
\renewcommand{\arraystretch}{1.75}
\begin{tabular}{|r|r|>{$}r<{$}>{$}r<{$}>{$}r<{$}||>{$}r<{$}|>{$}r<{$}>{$}r<{$}>{$}r<{$}>{$}r<{$}>{$}r<{$}|l}
  $i$ & $r_i$ & \text{entropy}_{0~0} & \text{entropy}_{0~1} & \text{entropy}_{0~2}  & u_i & \text{zeroes}_i & m_i & \text{offset}_i & \text{add}_i & \text{sign}_i \\
\cline{0-10}
-2 & & & & & \textit{und.} & & & & & \\
-1 & & {\color{red}0} & 0 & 0 & {\color{red}\textit{und.}} & \textit{und.} & \textit{und.} & \textit{und.} & \textit{und.} & \textit{und.} \\
0 & 1 & 0 & 0 & 0 & 3 & {\color{blue}0} & 1 & 0 & 0 & 0 \\
1 & 2 & 5 & 0 & 0 & 3 & \textit{und.} & 2 & 0 & 0 & 0 \\
2 & 3 & 10 & 5 & 0 & 5 & \textit{und.} & 3 & 0 & 0 & 0 \\
3 & 2 & 15 & 10 & 5 & 2 & \textit{und.} & 2 & 0 & 0 & 0 \\
4 & 1 & 20 & 15 & 3 & \textit{und.} & \textit{und.} & 0 & 1 & 1 & 0 \\
5 & 0 & 18 & 15 & 3 & 0 & \textit{und.} & 0 & 0 & 1 & 0 \\
6 & 0 & 16 & 15 & 3 & \textit{und.} & \textit{und.} & 0 & 0 & 1 & 0 \\
7 & 0 & 14 & 15 & 3 & 0 & \textit{und.} & 0 & 0 & 0 & 0 \\
8 & 0 & 12 & 15 & 3 & \textit{und.} & \textit{und.} & 0 & 0 & 0 & 0 \\
9 & 0 & 10 & 15 & 3 & 0 & \textit{und.} & 0 & 0 & 0 & 0 \\
10 & 0 & 8 & 15 & 3 & \textit{und.} & \textit{und.} & 0 & 0 & 0 & 0 \\
11 & 0 & 6 & 15 & 3 & 0 & \textit{und.} & 0 & 0 & 0 & 0 \\
12 & 0 & 4 & 15 & 3 & \textit{und.} & \textit{und.} & 0 & 0 & 0 & 0 \\
13 & 0 & 2 & 15 & 3 & 0 & \textit{und.} & 0 & 0 & 0 & 0 \\
14 & 0 & {\color{red}0} & 15 & 3 & {\color{red}\textit{und.}} & \textit{und.} & 0 & 0 & 0 & 0 \\
\cline{0-10}
15 & 0 & 0 & 15 & 3 & \textit{und.} & 1 & \textit{und.} & \textit{und.} & \textit{und.} & \textit{und.} & \multirow{10}{1em}{\begin{sideways}block of 10 zero residuals\end{sideways}} \\
16 & 0 & 0 & 0 & 0 & \textit{und.} & 2 & \textit{und.} & \textit{und.} & \textit{und.} & \textit{und.} \\
17 & 0 & 0 & 0 & 0 & \textit{und.} & 3 & \textit{und.} & \textit{und.} & \textit{und.} & \textit{und.} \\
18 & 0 & 0 & 0 & 0 & \textit{und.} & 4 & \textit{und.} & \textit{und.} & \textit{und.} & \textit{und.} \\
19 & 0 & 0 & 0 & 0 & \textit{und.} & 5 & \textit{und.} & \textit{und.} & \textit{und.} & \textit{und.} \\
20 & 0 & 0 & 0 & 0 & \textit{und.} & 6 & \textit{und.} & \textit{und.} & \textit{und.} & \textit{und.} \\
21 & 0 & 0 & 0 & 0 & \textit{und.} & 7 & \textit{und.} & \textit{und.} & \textit{und.} & \textit{und.} \\
22 & 0 & 0 & 0 & 0 & \textit{und.} & 8 & \textit{und.} & \textit{und.} & \textit{und.} & \textit{und.} \\
23 & 0 & 0 & 0 & 0 & \textit{und.} & 9 & \textit{und.} & \textit{und.} & \textit{und.} & \textit{und.} \\
24 & 0 & 0 & 0 & 0 & \textit{und.} & 10 & \textit{und.} & \textit{und.} & \textit{und.} & \textit{und.} \\
\cline{0-10}
25 & -1 & 0 & 0 & 0 & 1 & {\color{blue}10} & 0 & 0 & 0 & 1 \\
26 & -2 & 0 & 0 & 0 & 1 & \textit{und.} & 1 & 0 & 0 & 1 \\
27 & -3 & 5 & 0 & 0 & 3 & \textit{und.} & 2 & 0 & 0 & 1 \\
28 & -2 & 10 & 5 & 0 & 0 & \textit{und.} & 1 & 0 & 0 & 1 \\
29 & -1 & 15 & 3 & 0 & \textit{und.} & \textit{und.} & 0 & 0 & 0 & 1 \\
\cline{0-10}
\end{tabular}
}

\clearpage

For $r_0$, because $\text{entropy}_{0~0} = 0$ and
$u_{(-1)} = \textit{undefined}$\footnote{as determined by the \texttt{unary\_undefined} function},
we must handle a block of zeroes in some way.
But because $r_0 \neq 0$, we prepend a ``false alarm'' block of zeroes
and encode the residual normally.

For $r_{15}$, because $\text{entropy}_{0~0} = 0$,
$u_{14} = \textit{undefined}$ and $r_{15} = 0$,
we flush $\text{residual}_{14}$'s values and begin a block of zeroes
which continues until $r_{25} \neq 0$.
This block of zeroes is prepended to $\text{residual}_{25}$'s
values, which are flushed once $\text{residual}_{26}$ is encoded
and $u_{25}$ can be calculated from $u_{24}$ and $m_{26}$.

\begin{figure}[h]
  \includegraphics{wavpack/figures/residuals_parse2.pdf}
\end{figure}


\clearpage

\subsection{Writing RIFF WAVE Header and Footer}
\label{wavpack:write_wave_header}
\begin{figure}[h]
  \includegraphics{wavpack/figures/pcm_sandwich.pdf}
\end{figure}


\subsection{Writing MD5 Sum}
\label{wavpack:write_md5}
MD5 sum is calculated as if the PCM data had been read from
a wave file's \texttt{data} chunk.
That is, the samples are converted to little-endian format
and are signed if the stream's bits-per-sample is greater than 8.
\begin{figure}[h]
  \includegraphics{wavpack/figures/md5sum.pdf}
\end{figure}

\subsection{Writing Extended Integers}
\label{wavpack:write_extended_integers}
\begin{figure}[h]
  \includegraphics{wavpack/figures/extended_integers.pdf}
\end{figure}

\clearpage

\subsection{Writing Channel Info}
\label{wavpack:write_channel_info}
\begin{figure}[h]
  \includegraphics{wavpack/figures/channel_info.pdf}
\end{figure}


\subsection{Writing Sample Rate}
\label{wavpack:write_sample_rate}
\begin{figure}[h]
  \includegraphics{wavpack/figures/sample_rate.pdf}
\end{figure}

\clearpage

\subsection{Writing Block Header}
\label{wavpack:write_block_header}
{\relsize{-1}
  \input{wavpack/algorithms/write_block_header}
}

\clearpage

\subsection{Round-Tripping Correlation Weights}
\label{wavpack:roundtrip_weights}
Because the final weight values of one block
may not be exactly representable in a correlation weights sub-block,
it's necessary to ``round-trip'' the weight values
so that the starting values for the next block
are the same as the values stored in the sub-block.

\input{wavpack/algorithms/roundtrip_weights}

\clearpage

\subsection{Round-Tripping Correlation Samples}
\label{wavpack:roundtrip_samples}

\input{wavpack/algorithms/roundtrip_samples}
