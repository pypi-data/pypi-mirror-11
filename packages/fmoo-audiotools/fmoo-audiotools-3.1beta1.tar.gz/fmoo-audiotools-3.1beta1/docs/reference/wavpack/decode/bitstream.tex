%Copyright (C) 2007-2015  Brian Langenberger
%This work is licensed under the
%Creative Commons Attribution-Share Alike 3.0 United States License.
%To view a copy of this license, visit
%http://creativecommons.org/licenses/by-sa/3.0/us/ or send a letter to
%Creative Commons,
%171 Second Street, Suite 300,
%San Francisco, California, 94105, USA.

\subsection{Reading Bitstream}
\label{wavpack:read_bitstream}
{\relsize{-1}
  \input{wavpack/algorithms/read_bitstream}
}


\subsubsection{Reading Modified Elias Gamma Code}
\label{wavpack:read_egc}
{\relsize{-1}
  \input{wavpack/algorithms/read_egc}
}

\clearpage

\subsubsection{Reading Residual Value}
\label{wavpack:read_residual}
{\relsize{-1}
  \input{wavpack/algorithms/read_residual}
}

\clearpage

\subsubsection{Determining Base and Add Values}
\label{wavpack:decode_base_add}
{\relsize{-1}
  \input{wavpack/algorithms/decode_base_add}
}

\clearpage

\begin{figure}[h]
  \includegraphics{wavpack/figures/residuals.pdf}
\end{figure}

\subsubsection{Residual Decoding Example}
Given a 2 channel block with $\text{entropies}_0 = \texttt{[118, 194, 322]}$
and $\text{entropies}_1 = \texttt{[118, 176, 212]}$:

\begin{figure}[h]
\includegraphics[width=6in,keepaspectratio]{wavpack/figures/residuals_parse.pdf}
\end{figure}
\par
\noindent
Calculations of $\text{entropy}_0$ and $\text{entropy}_1$
are left as an exercise for the reader.

\clearpage

\begin{table}[h]
{\relsize{-3}
\begin{tabular}{|>{$}r<{$}||>{$}r<{$}|>{$}r<{$}|>{$}r<{$}|>{$}r<{$}|}
i & u_i & m_i &
\text{base} & \text{add} \\
%% \text{entropy}_{c~0} & \text{entropy}_{c~1} & \text{entropy}_{c~2} \\
\hline
0 &
7 &
\lfloor 7 \div 2 \rfloor = 3 &
2 + \left\lfloor\frac{118}{2 ^ 4}\right\rfloor + \left\lfloor\frac{194}{2 ^ 4}\right\rfloor + \left(\left\lfloor\frac{322}{2 ^ 4}\right\rfloor \times 1\right) = 42 & \lfloor 322 \div 2 ^ 4 \rfloor = 20 \\
%% 118 + 5 = 123 & 194 + 20 = 214 & 322 + 55 = 377
1 &
3 &
\lfloor 3 \div 2 \rfloor + 1 = 2 &
2 + \left\lfloor\frac{118}{2 ^ 4}\right\rfloor + \left\lfloor\frac{176}{2 ^ 4}\right\rfloor = 20 & \lfloor 212 \div 2 ^ 4 \rfloor = 13 \\
%% 118 + 5 = 123 & 176 + 15 = 191 & 212 - 35 = 198
\hline
2 &
3 &
\lfloor 3 \div 2 \rfloor + 1 = 2 &
2 + \left\lfloor\frac{123}{2 ^ 4}\right\rfloor + \left\lfloor\frac{214}{2 ^ 4}\right\rfloor = 22 & \lfloor 377 \div 2 ^ 4 \rfloor = 23 \\
%% 123 + 5 = 128 & 214 + 20 = 234 & 377 - 60 = 353
3 &
3 &
\lfloor 3 \div 2 \rfloor + 1 = 2 &
2 + \left\lfloor\frac{123}{2 ^ 4}\right\rfloor + \left\lfloor\frac{191}{2 ^ 4}\right\rfloor = 20 & \lfloor 198 \div 2 ^ 4 \rfloor = 12 \\
%% 123 + 5 = 128 & 191 + 15 = 206 & 198 - 35 = 184
\hline
4 &
1 &
\lfloor 1 \div 2 \rfloor + 1 = 1 &
1 + \left\lfloor\frac{128}{2 ^ 4}\right\rfloor = 9 & \lfloor 234 \div 2 ^ 4 \rfloor = 14 \\
%% 128 + 10 = 138 & 234 - 8 = 226 & 353
5 &
4 &
\lfloor 4 \div 2 \rfloor + 1 = 3 &
2 + \left\lfloor\frac{128}{2 ^ 4}\right\rfloor + \left\lfloor\frac{206}{2 ^ 4}\right\rfloor + \left(\left\lfloor\frac{184}{2 ^ 4}\right\rfloor \times 1\right) = 34 & \lfloor 184 \div 2 ^ 4 \rfloor = 11 \\
%% 128 + 10 = 138 & 206 + 20 = 226 & 184 + 30 = 214
\hline
6 &
\textit{undefined} &
0 &
0 & \lfloor 138 \div 2 ^ 4 \rfloor = 8 \\
%% 138 - 4 = 134 & 226 & 353
7 &
5 &
\lfloor 5 \div 2 \rfloor = 2 &
2 + \left\lfloor\frac{138}{2 ^ 4}\right\rfloor + \left\lfloor\frac{226}{2 ^ 4}\right\rfloor = 24 & \lfloor 214 \div 2 ^ 4 \rfloor = 13 \\
%% 138 + 10 = 148 & 226 + 20 = 246 & 214 - 35 = 200
\hline
8 &
1 &
\lfloor 1 \div 2 \rfloor + 1 = 1 &
1 + \left\lfloor\frac{134}{2 ^ 4}\right\rfloor = 9 & \lfloor 226 \div 2 ^ 4 \rfloor = 14 \\
%% 134 + 10 = 144 & 226 - 8 = 218 & 353
9 &
3 &
\lfloor 3 \div 2 \rfloor + 1 = 2 &
2 + \left\lfloor\frac{148}{2 ^ 4}\right\rfloor + \left\lfloor\frac{246}{2 ^ 4}\right\rfloor = 26 & \lfloor 200 \div 2 ^ 4 \rfloor = 12 \\
%% 148 + 10 = 158 & 246 + 20 = 266 & 200 - 35 = 186
\hline
10 &
3 &
\lfloor 3 \div 2 \rfloor + 1 = 2 &
2 + \left\lfloor\frac{144}{2 ^ 4}\right\rfloor + \left\lfloor\frac{218}{2 ^ 4}\right\rfloor = 24 & \lfloor 353 \div 2 ^ 4 \rfloor = 22 \\
%% 144 + 10 = 154 & 218 + 20 = 238 & 353 - 55 = 331
11 &
3 &
\lfloor 3 \div 2 \rfloor + 1 = 2 &
2 + \left\lfloor\frac{158}{2 ^ 4}\right\rfloor + \left\lfloor\frac{266}{2 ^ 4}\right\rfloor = 27 & \lfloor 186 \div 2 ^ 4 \rfloor = 11 \\
%% 158 + 10 = 168 & 266 + 25 = 291 & 186 - 30 = 174
\hline
12 &
5 &
\lfloor 5 \div 2 \rfloor + 1 = 3 &
2 + \left\lfloor\frac{154}{2 ^ 4}\right\rfloor + \left\lfloor\frac{238}{2 ^ 4}\right\rfloor + \left(\left\lfloor\frac{331}{2 ^ 4}\right\rfloor \times 1\right) = 46 & \lfloor 331 \div 2 ^ 4 \rfloor = 20 \\
%% 154 + 10 = 164 & 238 + 20 = 258 & 331 + 55 = 386
13 &
1 &
\lfloor 1 \div 2 \rfloor + 1 = 1 &
1 + \left\lfloor\frac{168}{2 ^ 4}\right\rfloor = 11 & \lfloor 291 \div 2 ^ 4 \rfloor = 18 \\
%% 168 + 10 = 178 & 291 - 10 = 281 & 174
\hline
14 &
5 &
\lfloor 5 \div 2 \rfloor + 1 = 3 &
2 + \left\lfloor\frac{164}{2 ^ 4}\right\rfloor + \left\lfloor\frac{258}{2 ^ 4}\right\rfloor + \left(\left\lfloor\frac{386}{2 ^ 4}\right\rfloor \times 1\right) = 53 & \lfloor 386 \div 2 ^ 4 \rfloor = 24 \\
%% 164 + 10 = 174 & 258 + 25 = 283 & 386 + 65 = 451
15 &
1 &
\lfloor 1 \div 2 \rfloor + 1 = 1 &
1 + \left\lfloor\frac{178}{2 ^ 4}\right\rfloor = 12 & \lfloor 281 \div 2 ^ 4 \rfloor = 17 \\
%% 178 + 10 = 188 & 281 - 10 = 271 & 174
\hline
16 &
4 &
\lfloor 4 \div 2 \rfloor + 1 = 3 &
2 + \left\lfloor\frac{174}{2 ^ 4}\right\rfloor + \left\lfloor\frac{283}{2 ^ 4}\right\rfloor + \left(\left\lfloor\frac{451}{2 ^ 4}\right\rfloor \times 1\right) = 58 & \lfloor 451 \div 2 ^ 4 \rfloor = 28 \\
%% 174 + 10 = 184 & 283 + 25 = 308 & 451 + 75 = 526
17 &
\textit{undefined} &
0 &
0 & \lfloor 188 \div 2 ^ 4 \rfloor = 11 \\
%% 188 - 4 = 184 & 271 & 174
\hline
18 &
6 &
\lfloor 6 \div 2 \rfloor = 3 &
2 + \left\lfloor\frac{184}{2 ^ 4}\right\rfloor + \left\lfloor\frac{308}{2 ^ 4}\right\rfloor + \left(\left\lfloor\frac{526}{2 ^ 4}\right\rfloor \times 1\right) = 65 & \lfloor 526 \div 2 ^ 4 \rfloor = 32 \\
%% 184 + 10 = 194 & 308 + 25 = 333 & 526 + 85 = 611
19 &
\textit{undefined} &
0 &
0 & \lfloor 184 \div 2 ^ 4 \rfloor = 11 \\
%% 184 - 4 = 180 & 271 & 174
\hline
\end{tabular}
}
\vskip .25in
{\relsize{-3}
\renewcommand{\arraystretch}{1.5}
\begin{tabular}{|>{$}r<{$}|>{$}r<{$}|>{$}r<{$}||>{$}r<{$}|>{$}r<{$}|>{$}r<{$}|>{$}r<{$}|>{$}r<{$}|>{$}r<{$}|>{$}r<{$}|}
i & \text{base} & \text{add} & p & e & r_i & b_i & unsigned & sign_i & residual_i \\
\hline
0 &
42 & 20 &
\lfloor\log_2(20)\rfloor = 4 &
2 ^ {4 + 1} - 20 - 1 = 11 &
14 &
1 & 42 + (14 \times 2) - 11 + 1 = 60 &
1 & -60 - 1 = -61
\\
1 &
20 & 13 &
\lfloor\log_2(13)\rfloor = 3 &
2 ^ {3 + 1} - 13 - 1 = 2 &
6 &
1 & 20 + (6 \times 2) - 2 + 1 = 31 &
0 & 31
\\
\hline
2 &
22 & 23 &
\lfloor\log_2(23)\rfloor = 4 &
2 ^ {4 + 1} - 23 - 1 = 8 &
9 &
0 & 22 + (9 \times 2) - 8 + 0 = 32 &
1 & -32 - 1 = -33
\\
3 &
20 & 12 &
\lfloor\log_2(12)\rfloor = 3 &
2 ^ {3 + 1} - 12 - 1 = 3 &
7 &
1 & 20 + (7 \times 2) - 3 + 1 = 32 &
0 & 32
\\
\hline
4 &
9 & 14 &
\lfloor\log_2(14)\rfloor = 3 &
2 ^ {3 + 1} - 14 - 1 = 1 &
4 &
1 & 9 + (4 \times 2) - 1 + 1 = 17 &
1 & -17 - 1 = -18
\\
5 &
34 & 11 &
\lfloor\log_2(11)\rfloor = 3 &
2 ^ {3 + 1} - 11 - 1 = 4 &
2 &
 & 34 + 2 = 36 &
0 & 36
\\
\hline
6 &
0 & 8 &
\lfloor\log_2(8)\rfloor = 3 &
2 ^ {3 + 1} - 8 - 1 = 7 &
1 &
 & 0 + 1 = 1 &
0 & 1
\\
7 &
24 & 13 &
\lfloor\log_2(13)\rfloor = 3 &
2 ^ {3 + 1} - 13 - 1 = 2 &
7 &
1 & 24 + (7 \times 2) - 2 + 1 = 37 &
0 & 37
\\
\hline
8 &
9 & 14 &
\lfloor\log_2(14)\rfloor = 3 &
2 ^ {3 + 1} - 14 - 1 = 1 &
6 &
0 & 9 + (6 \times 2) - 1 + 0 = 20 &
0 & 20
\\
9 &
26 & 12 &
\lfloor\log_2(12)\rfloor = 3 &
2 ^ {3 + 1} - 12 - 1 = 3 &
6 &
0 & 26 + (6 \times 2) - 3 + 0 = 35 &
0 & 35
\\
\hline
10 &
24 & 22 &
\lfloor\log_2(22)\rfloor = 4 &
2 ^ {4 + 1} - 22 - 1 = 9 &
10 &
0 & 24 + (10 \times 2) - 9 + 0 = 35 &
0 & 35
\\
11 &
27 & 11 &
\lfloor\log_2(11)\rfloor = 3 &
2 ^ {3 + 1} - 11 - 1 = 4 &
4 &
0 & 27 + (4 \times 2) - 4 + 0 = 31 &
0 & 31
\\
\hline
12 &
46 & 20 &
\lfloor\log_2(20)\rfloor = 4 &
2 ^ {4 + 1} - 20 - 1 = 11 &
4 &
 & 46 + 4 = 50 &
0 & 50
\\
13 &
11 & 18 &
\lfloor\log_2(18)\rfloor = 4 &
2 ^ {4 + 1} - 18 - 1 = 13 &
13 &
1 & 11 + (13 \times 2) - 13 + 1 = 25 &
0 & 25
\\
\hline
14 &
53 & 24 &
\lfloor\log_2(24)\rfloor = 4 &
2 ^ {4 + 1} - 24 - 1 = 7 &
8 &
0 & 53 + (8 \times 2) - 7 + 0 = 62 &
0 & 62
\\
15 &
12 & 17 &
\lfloor\log_2(17)\rfloor = 4 &
2 ^ {4 + 1} - 17 - 1 = 14 &
6 &
 & 12 + 6 = 18 &
0 & 18
\\
\hline
16 &
58 & 28 &
\lfloor\log_2(28)\rfloor = 4 &
2 ^ {4 + 1} - 28 - 1 = 3 &
6 &
1 & 58 + (6 \times 2) - 3 + 1 = 68 &
0 & 68
\\
17 &
0 & 11 &
\lfloor\log_2(11)\rfloor = 3 &
2 ^ {3 + 1} - 11 - 1 = 4 &
7 &
0 & 0 + (7 \times 2) - 4 + 0 = 10 &
0 & 10
\\
\hline
18 &
65 & 32 &
\lfloor\log_2(32)\rfloor = 5 &
2 ^ {5 + 1} - 32 - 1 = 31 &
6 &
 & 65 + 6 = 71 &
0 & 71
\\
19 &
0 & 11 &
\lfloor\log_2(11)\rfloor = 3 &
2 ^ {3 + 1} - 11 - 1 = 4 &
0 &
 & 0 + 0 = 0 &
0 & 0
\\
\hline
\end{tabular}
}
{\relsize{-1}
  \vskip .2in
  Resulting in:
  \newline
  \begin{tabular}{rr}
    channel 0 residuals : & \texttt{[-61,~-33,~-18,~~1,~20,~35,~50,~62,~68,~71]}\\
    channel 1 residuals : & \texttt{[~31,~~32,~~36,~37,~35,~31,~25,~18,~10,~~0]}\\
  \end{tabular}
}
\end{table}

\clearpage

\subsubsection{2nd Residual Decoding Example}
Given a 1 channel block with $\text{entropies}_0 = \texttt{[0, 0, 0]}$:

\begin{figure}[h]
\includegraphics{wavpack/figures/residuals_parse2.pdf}
\end{figure}
\par
\noindent
This residual block demonstrates how a long run of 0 residuals is read
when $\text{entropy}_{0~0}$ falls to a low enough value.
Calculation of $\text{entropies}_{0~1}$ and $\text{entropies}_{0~2}$
are left as an exercise for the reader.

\clearpage

\begin{table}[h]
{\relsize{-3}
\renewcommand{\arraystretch}{1.5}
\begin{tabular}{|>{$}r<{$}||>{$}r<{$}|>{$}r<{$}|>{$}r<{$}|>{$}r<{$}||>{$}r<{$}|>{$}r<{$}|>{$}r<{$}|}
i & u_i & m_i &
\text{base} & \text{add} & \text{entropy}_{0~0} \\
\hline
& \multicolumn{5}{c|}{read modified Elias gamma code: $t \leftarrow 0~,~\text{zeroes} \leftarrow 0$} \\
0 &
3 & \lfloor 3 \div 2\rfloor = 1 &
1 + \left\lfloor\frac{0}{2 ^ 4}\right\rfloor = 1 &
\left\lfloor\frac{0}{2 ^ 4}\right\rfloor = 0 &
0 + 5 = 5 %% & 0 - 0 = 0 & 0
\\
1 &
3 & \lfloor 3 \div 2\rfloor + 1 = 2 &
2 + \left\lfloor\frac{5}{2 ^ 4}\right\rfloor + \left\lfloor\frac{0}{2 ^ 4}\right\rfloor = 2 &
\left\lfloor\frac{0}{2 ^ 4}\right\rfloor = 0 &
5 + 5 = 10 %% & 0 + 5 = 5 & 0 - 0 = 0
\\
2 &
5 & \lfloor 5 \div 2\rfloor + 1 = 3 &
2 + \left\lfloor\frac{10}{2 ^ 4}\right\rfloor + \left\lfloor\frac{5}{2 ^ 4}\right\rfloor + \left(\left\lfloor\frac{0}{2 ^ 4}\right\rfloor \times 1\right) = 3 &
\left\lfloor\frac{0}{2 ^ 4}\right\rfloor = 0 &
10 + 5 = 15 %% & 5 + 5 = 10 & 0 + 5 = 5
\\
3 &
2 & \lfloor 2 \div 2\rfloor + 1 = 2 &
2 + \left\lfloor\frac{15}{2 ^ 4}\right\rfloor + \left\lfloor\frac{10}{2 ^ 4}\right\rfloor = 2 &
\left\lfloor\frac{5}{2 ^ 4}\right\rfloor = 0 &
15 + 5 = 20 %% & 10 + 5 = 15 & 5 - 2 = 3
\\
4 &
\textit{undefined} & 0 &
0 & \left\lfloor\frac{20}{2 ^ 4}\right\rfloor = 1 &
20 - 2 = 18 %% & 15 & 3
\\
5 &
0 & \lfloor 0 \div 2\rfloor = 0 &
0 & \left\lfloor\frac{18}{2 ^ 4}\right\rfloor = 1 &
18 - 2 = 16 %% & 15 & 3
\\
6 &
\textit{undefined} & 0 &
0 & \left\lfloor\frac{16}{2 ^ 4}\right\rfloor = 1 &
16 - 2 = 14 %% & 15 & 3
\\
7 &
0 & \lfloor 0 \div 2\rfloor = 0 &
0 & \left\lfloor\frac{14}{2 ^ 4}\right\rfloor = 0 &
14 - 2 = 12 %% & 15 & 3
\\
8 &
\textit{undefined} & 0 &
0 & \left\lfloor\frac{12}{2 ^ 4}\right\rfloor = 0 &
12 - 2 = 10 %% & 15 & 3
\\
9 &
0 & \lfloor 0 \div 2\rfloor = 0 &
0 & \left\lfloor\frac{10}{2 ^ 4}\right\rfloor = 0 &
10 - 2 = 8 %% & 15 & 3
\\
10 &
\textit{undefined} & 0 &
0 & \left\lfloor\frac{8}{2 ^ 4}\right\rfloor = 0 &
8 - 2 = 6 %% & 15 & 3
\\
11 &
0 & \lfloor 0 \div 2\rfloor = 0 &
0 & \left\lfloor\frac{6}{2 ^ 4}\right\rfloor = 0 &
6 - 2 = 4 %% & 15 & 3
\\
12 &
\textit{undefined} & 0 &
0 & \left\lfloor\frac{4}{2 ^ 4}\right\rfloor = 0 &
4 - 2 = 2 %% & 15 & 3
\\
13 &
0 & \lfloor 0 \div 2\rfloor = 0 &
0 & \left\lfloor\frac{2}{2 ^ 4}\right\rfloor = 0 &
2 - 2 = 0 %% & 15 & 3
\\
14 &
\textit{\color{red}undefined} & 0 &
0 & \left\lfloor\frac{0}{2 ^ 4}\right\rfloor = 0 &
0 - 0 = {\color{red}0} %% & 15 & 3
\\
15-24 & \multicolumn{5}{c|}{read modified Elias gamma code: $t \leftarrow 4~,~p \leftarrow 2~,~\text{zeroes} \leftarrow 2 ^ {(4 - 1)} + 2 = 10$} \\
25 &
1 & \lfloor 1 \div 2\rfloor = 0 &
0 & \left\lfloor\frac{0}{2 ^ 4}\right\rfloor = 0 &
0 - 0 = 0 %% & 0 & 0
\\
26 &
1 & \lfloor 1 \div 2\rfloor + 1 = 1 &
1 + \left\lfloor\frac{0}{2 ^ 4}\right\rfloor = 1 &
\left\lfloor\frac{0}{2 ^ 4}\right\rfloor = 0 &
0 + 5 = 5 %% & 0 - 0 = 0 & 0
\\
27 &
3 & \lfloor 3 \div 2\rfloor + 1 = 2 &
2 + \left\lfloor\frac{5}{2 ^ 4}\right\rfloor + \left\lfloor\frac{0}{2 ^ 4}\right\rfloor = 2 &
\left\lfloor\frac{0}{2 ^ 4}\right\rfloor = 0 &
5 + 5 = 10 %% & 0 + 5 = 5 & 0 - 0 = 0
\\
28 &
0 & \lfloor 0 \div 2\rfloor + 1 = 1 &
1 + \left\lfloor\frac{10}{2 ^ 4}\right\rfloor = 1 &
\left\lfloor\frac{5}{2 ^ 4}\right\rfloor = 0 &
10 + 5 = 15 %% & 5 - 2 = 3 & 0
\\
29 &
\textit{undefined} & 0 &
0 & \left\lfloor\frac{15}{2 ^ 4}\right\rfloor = 0 &
15 - 2 = 13 %% & 3 & 0
\\
\hline
\end{tabular}
}
\vskip .25in
{\relsize{-3}
\begin{tabular}{|>{$}r<{$}|>{$}r<{$}|>{$}r<{$}||>{$}r<{$}|>{$}r<{$}|>{$}r<{$}|>{$}r<{$}|>{$}r<{$}|>{$}r<{$}|>{$}r<{$}|}
i & \text{base} & \text{add} & p & e & r_i & b_i & unsigned & sign_i & residual_i \\
\hline
0 &
1 & 0 &
& & & & 1 &
0 & 1
\\
1 &
2 & 0 &
& & & & 2 &
0 & 2
\\
2 &
3 & 0 &
& & & & 3 &
0 & 3
\\
3 &
2 & 0 &
& & & & 2 &
0 & 2
\\
4 &
0 & 1 &
\lfloor\log_2(1)\rfloor = 0 &
2 ^ {0 + 1} - 1 - 1 = 0 &
0 &
1 & 0 + (0 \times 2) - 0 + 1 = 1 &
0 & 1
\\
5 &
0 & 1 &
\lfloor\log_2(1)\rfloor = 0 &
2 ^ {0 + 1} - 1 - 1 = 0 &
0 &
0 & 0 + (0 \times 2) - 0 + 0 = 0 &
0 & 0
\\
6 &
0 & 1 &
\lfloor\log_2(1)\rfloor = 0 &
2 ^ {0 + 1} - 1 - 1 = 0 &
0 &
0 & 0 + (0 \times 2) - 0 + 0 = 0 &
0 & 0
\\
7 &
0 & 0 &
& & & & 0 &
0 & 0
\\
8 &
0 & 0 &
& & & & 0 &
0 & 0
\\
9 &
0 & 0 &
& & & & 0 &
0 & 0
\\
10 &
0 & 0 &
& & & & 0 &
0 & 0
\\
11 &
0 & 0 &
& & & & 0 &
0 & 0
\\
12 &
0 & 0 &
& & & & 0 &
0 & 0
\\
13 &
0 & 0 &
& & & & 0 &
0 & 0
\\
14 &
0 & 0 &
& & & & 0 &
0 & 0
\\
15-24 & \multicolumn{8}{c|}{long run of 10, 0 residual values} & 0 \\
25 &
0 & 0 &
& & & & 0 &
1 & -0 - 1 = -1
\\
26 &
1 & 0 &
& & & & 1 &
1 & -1 - 1 = -2
\\
27 &
2 & 0 &
& & & & 2 &
1 & -2 - 1 = -3
\\
28 &
1 & 0 &
& & & & 1 &
1 & -1 - 1 = -2
\\
29 &
0 & 0 &
& & & & 0 &
1 & -0 - 1 = -1
\\
\hline
\end{tabular}
}
{\relsize{-3}
  \vskip .2in
    Resulting in channel 0 residuals:
    \newline
    \texttt{[~~1,~~2,~~3,~~2,~~1,~~0,~~0,~~0,~~0,~~0,~~0,~~0,~~0,~~0,~~0,~~0,~~0,~~0,~~0,~~0,~~0,~~0,~~0,~~0,~~0,~-1,~-2,~-3,~-2,~-1]}
}
\end{table}
